\chapter{Structured Matrix Types}

\NOTE{Structured matrix types are experimental in {\ViennaCLversion}. Interface changes as well as considerable performance improvements may be included in
future releases!}

There are a number of structured dense matrices for which some algorithms such as matrix-vector products can be computed with much lower computational effort
than for the general dense matrix case. In the following, four structured dense matrix types included in {\ViennaCL} are discussed.
Example code can be found in \lstinline|examples/tutorial/structured-matrices.cpp|.

\section{Circulant Matrix}
A circulant matrix is a matrix of the form
\begin{align*}
 \left( \begin{array}{ccccc}
         c_0 & c_{n-1} & \ldots & c_2 & c_1 \\
         c_1 & c_0 & c_{n-1} & & c_2 \\
         \vdots & c_1 & c_0 & \ddots & \vdots \\
         c_{n-2} & & \ddots & \ddots & c_{n-1} \\
         c_{n-1} & c_{n-2} & \hdots & c_1 & c_0 \\
        \end{array} \right)
\end{align*}
and available in {\ViennaCL} via
\begin{lstlisting}
 #include "viennacl/circulant_matrix.hpp"

 std::size_t s = 42;
 viennacl::circulant_matrix circ_mat(s, s);
\end{lstlisting}
The \lstinline|circulant_matrix| type can be manipulated in the same way as the dense matrix type \lstinline|matrix|. Note that writing to a single element of
the matrix is structure-preserving, e.g.~changing \lstinline|circ_mat(1,2)| will automatically update \lstinline|circ_mat(0,1)|, \lstinline|circ_mat(2,3)| and
so on.


\section{Hankel Matrix}
A Hankel matrix is a matrix of the form
\begin{align*}
 \left( \begin{array}{cccc}
         a & b & c & d \\
         b & c & d & e \\
         c & d & e & f \\
         d & e & f & g \\
        \end{array} \right)
\end{align*}
and available in {\ViennaCL} via
\begin{lstlisting}
 #include "viennacl/hankel_matrix.hpp"

 std::size_t s = 42;
 viennacl::hankel_matrix hank_mat(s, s);
\end{lstlisting}
The \lstinline|hankel_matrix| type can be manipulated in the same way as the dense matrix type \lstinline|matrix|. Note that writing to a single element of
the matrix is structure-preserving, e.g.~changing \lstinline|hank_mat(1,2)| in the example above will also update \lstinline|hank_mat(0,3)|,
\lstinline|hank_mat(2,1)| and
\lstinline|hank_mat(3,0)|.

\section{Toeplitz Matrix}
A Toeplitz matrix is a matrix of the form
\begin{align*}
 \left( \begin{array}{cccc}
         a & b & c & d \\
         e & a & b & c \\
         f & e & a & b \\
         g & f & e & a \\
        \end{array} \right)
\end{align*}
and available in {\ViennaCL} via
\begin{lstlisting}
 #include "viennacl/toeplitz_matrix.hpp"

 std::size_t s = 42;
 viennacl::toeplitz_matrix toep_mat(s, s);
\end{lstlisting}
The \lstinline|toeplitz_matrix| type can be manipulated in the same way as the dense matrix type \lstinline|matrix|. Note that writing to a single element of
the matrix is structure-preserving, e.g.~changing \lstinline|toep_mat(1,2)| in the example above will also update \lstinline|toep_mat(0,1)| and
\lstinline|toep_mat(2,3)|.


\section{Vandermonde Matrix}
A Vandermonde matrix is a matrix of the form
\begin{align*}
 \left( \begin{array}{ccccc}
         1 & \alpha_1 & \alpha_1^2 & \ldots & \alpha_1^{n-1} \\
         1 & \alpha_2 & \alpha_2^2 & \ldots & \alpha_2^{n-1} \\
         1 & \vdots & \vdots & \vdots \\
         1 & \alpha_m & \alpha_m^2 & \ldots & \alpha_m^{n-1} \\
        \end{array} \right)
\end{align*}
and available in {\ViennaCL} via
\begin{lstlisting}
 #include "viennacl/vandermonde_matrix.hpp"

 std::size_t s = 42;
 viennacl::vandermonde_matrix vand_mat(s, s);
\end{lstlisting}
The \lstinline|vandermonde_matrix| type can be manipulated in the same way as the dense matrix type \lstinline|matrix|, but restrictions apply. For
example, the addition or subtraction of two Vandermonde matrices does not yield another Vandermonde matrix. Note that writing to a single element of
the matrix is structure-preserving, e.g.~changing \lstinline|vand_mat(1,2)| in the example above will automatically update \lstinline|vand_mat(1,3)|,
\lstinline|vand_mat(1,4)|, etc.

