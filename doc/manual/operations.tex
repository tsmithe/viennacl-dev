\chapter{Basic Operations} \label{chap:operations}

The basic types have been introduced in the previous chapter, so we move on with the description of the basic BLAS operations.
Almost all operations supported by {\ublas} are available, including element-wise operations on vectors. Thus, consider the
\href{http://www.boost.org/doc/libs/1_52_0/libs/numeric/ublas/doc/operations_overview.htm}{ublas-documentation} as a reference as well.

\section{Vector-Vector Operations (BLAS Level 1)}

{\ViennaCL} provides all vector-vector operations defined at level 1 of BLAS. Tab.~\ref{tab:blas-level-1} shows how these operations can be carried
out in \ViennaCL. The function interface is compatible with {\ublas},
thus allowing quick code migration for {\ublas} users.


\TIP{For full details on level 1 functions, refer to the reference documentation
located in \texttt{doc/doxygen/}}


\begin{table}[tb]
\begin{center}
\begin{tabular}{l|l|p{6cm}}
Verbal & Mathematics & ViennaCL\\
\hline
swap    & $x \leftrightarrow y$ & \lstinline|swap(x,y);| \\
stretch    & $x \leftarrow \alpha x$ & \lstinline|x *= alpha;| \\
assignment & $y \leftarrow x$ & \lstinline|y = x;| \\
multiply add & $y \leftarrow \alpha x + y$ & \lstinline|y += alpha * x;| \\
multiply subtract & $y \leftarrow \alpha x - y$ & \lstinline|y -= alpha * x;| \\
elementwise product & $y_i \leftarrow x_i \cdot z_i$ & \lstinline|y = element_prod(x,z);| \\
elementwise division & $y_i \leftarrow x_i \cdot z_i$ & \lstinline|y = element_div(x,z);| \\
inner dot product & $\alpha \leftarrow x^{\mathrm{T}} y$ & \lstinline|inner_prod(x,y);| \\
$L^1$ norm & $\alpha \leftarrow \Vert x \Vert_1$ & \lstinline|alpha = norm_1(x);| \\
$L^2$ norm & $\alpha \leftarrow \Vert x \Vert_2$ & \lstinline|alpha = norm_2(x);| \\
$L^\infty$ norm & $\alpha \leftarrow \Vert x \Vert_\infty$ & \lstinline|alpha = norm_inf(x);| \\
$L^\infty$ norm index& $i \leftarrow \max_i \vert x_i \vert$ & \lstinline|i = index_norm_inf(x);| \\
plane rotation & $(x,y) \leftarrow (\alpha x + \beta y, -\beta x + \alpha y)$ & \lstinline|plane_rotation(alpha, beta, x, y);| \\
\end{tabular}
\caption{BLAS level 1 routines mapped to {\ViennaCL}. Note that the free functions reside in namespace \texttt{viennacl::linalg}}
\label{tab:blas-level-1}
\end{center}
\end{table}

\section{Matrix-Vector Operations (BLAS Level 2)}
The interface for level 2 BLAS functions in {\ViennaCL} is similar to that of
{\ublas} and shown in Tab.~\ref{tab:blas-level-2}.

\TIP{For full details on level 2 functions, refer to the reference documentation
located in \texttt{doc/doxygen/}}


\begin{table}[tb]
\begin{center}
\renewcommand{\arraystretch}{1.2}
\begin{tabular}{p{4cm}|l|p{7cm}}
Verbal & Mathematics & ViennaCL\\
\hline
matrix vector product & $y \leftarrow A x$ & \lstinline|y = prod(A, x);| \\
matrix vector product & $y \leftarrow A^\mathrm{T} x$ & \lstinline|y = prod(trans(A), x);| \\
inplace mv product & $x \leftarrow A x$ & \lstinline|x = prod(A, x);| \\
inplace mv product & $x \leftarrow A^\mathrm{T} x$ & \lstinline|x = prod(trans(A), x);| \\
\hline
scaled product add & $y \leftarrow \alpha A x + \beta y$ & \lstinline|y = alpha * prod(A, x) + beta * y| \\
scaled product add & $y \leftarrow \alpha A^{\mathrm T} x + \beta y$ & \lstinline|y = alpha * prod(trans(A), x) + beta * y| \\
\hline
tri. matrix solve & $y \leftarrow A^{-1} x$ & \lstinline|y = solve(A, x, tag);| \\
tri. matrix solve & $y \leftarrow A^\mathrm{T^{-1}} x$ & \lstinline|y = solve(trans(A), x, tag);| \\
inplace solve & $x \leftarrow A^{-1} x$ & \lstinline|inplace_solve(A, x, tag);| \\
inplace solve & $x \leftarrow A^\mathrm{T^{-1}} x$ & \lstinline|inplace_solve(trans(A), x, tag);| \\
\hline
rank 1 update & $A \leftarrow \alpha x y^{\mathrm T} + A$ & \lstinline|A += alpha * outer_prod(x,y);| \\
symm. rank 1 update & $A \leftarrow \alpha x x^{\mathrm T} + A$ & \lstinline|A += alpha * outer_prod(x,x);| \\
rank 2 update & $A \leftarrow \alpha (x y^{\mathrm T} + y x^{\mathrm T}) + A$ & \lstinline|A += alpha * outer_prod(x,y);| \lstinline|A += alpha * outer_prod(y,x);| \\
\end{tabular}
\caption{BLAS level 2 routines mapped to \ViennaCL. Note that the free functions reside in namespace \texttt{viennacl::linalg}. \lstinline|tag| is one out of \lstinline|lower_tag|, \lstinline|unit_lower_tag|, \lstinline|upper_tag|, and \lstinline|unit_upper_tag|.}
\label{tab:blas-level-2}
\end{center}
\end{table}

\section{Matrix-Matrix Operations (BLAS Level 3)}
Full BLAS level 3 support is since {\ViennaCL} 1.1.0, cf.~Tab.~\ref{tab:blas-level-3}. While BLAS
levels 1 and 2 are mostly memory-bandwidth-limited, BLAS level 3 is mostly
limited by the available computational power of the respective device. Hence,
matrix-matrix products regularly show impressive performance gains on mid-
to high-end GPUs when compared to a single CPU core.

Again, the {\ViennaCL} API is identical to that of {\ublas} and comparisons can
be carried out immediately, as is shown in the tutorial located in
\texttt{examples/tutorial/blas3.cpp}.

As for performance, {\ViennaCL} yields decent performance gains at BLAS level
3 on mid- to high-end GPUs compared to CPU implementations using a single core
only. However, highest performance is usually obtained only with careful tuning to the respective target device.
Generally, {\ViennaCL} provides
kernels that represent a good compromise between efficiency and portability
among a large number of different devices and device types.

\TIP{
For certain matrix dimensions, typically multiples of 64 or 128, {\ViennaCL} also provides tuned kernels reaching over 1 TFLOP in single precision (AMD HD 7970).
}

\begin{table}[tb]
\begin{center}
\renewcommand{\arraystretch}{1.2}
\begin{tabular}{p{4cm}|l|p{7.5cm}}
Verbal & Mathematics & ViennaCL\\
\hline
matrix-matrix product & $C \leftarrow A \times B$ & \lstinline|C = prod(A, B);| \\
matrix-matrix product & $C \leftarrow A \times B^\mathrm{T}$ & \lstinline|C = prod(A, trans(B));| \\
matrix-matrix product & $C \leftarrow A^\mathrm{T} \times B$ & \lstinline|C = prod(trans(A), B);| \\
matrix-matrix product & $C \leftarrow A^\mathrm{T} \times B^\mathrm{T}$ & \lstinline|C = prod(trans(A), trans(B));| \\
\hline
tri. matrix solve & $C \leftarrow A^{-1} B$ & \lstinline|C = solve(A, B, tag);| \\
tri. matrix solve & $C \leftarrow A^\mathrm{T^{-1}} B$ & \lstinline|C = solve(trans(A), B, tag);| \\
tri. matrix solve & $C \leftarrow A^{-1} B^\mathrm{T}$ & \lstinline|C = solve(A, trans(B), tag);| \\
tri. matrix solve & $C \leftarrow A^\mathrm{T^{-1}} B^\mathrm{T}$ & \lstinline|C = solve(trans(A), trans(B), tag);| \\
%
inplace solve & $B \leftarrow A^{-1} B$ & \lstinline|inplace_solve(A, trans(B), tag);| \\
inplace solve & $B \leftarrow A^\mathrm{T^{-1}} B$ & \lstinline|inplace_solve(trans(A), x, tag);| \\
inplace solve & $B \leftarrow A^{-1} B^\mathrm{T}$ & \lstinline|inplace_solve(A, trans(B), tag);| \\
inplace solve & $B \leftarrow A^\mathrm{T^{-1}} B^\mathrm{T}$ & \lstinline|inplace_solve(trans(A), x, tag);| \\
\end{tabular}
\caption{BLAS level 3 routines mapped to \ViennaCL. Note that the free functions
reside in namespace \texttt{viennacl::linalg}}
\label{tab:blas-level-3}
\end{center}
\end{table}

\section{Initializer Types}

\NOTE{Initializer types in {\ViennaCLversion} can only be used for initializing vectors and matrices, not for computations!}

In order to initialize vectors, the following initializer types are provided, again similar to {\ublas}:
\begin{center}
\begin{tabular}{|l|p{10cm}|}
 \hline
 \lstinline|unit_vector<T>(s, i)| & Unit vector of size $s$ with entry $1$ at index $i$, zero elsewhere. \\
 \hline
 \lstinline|zero_vector<T>(s)| & Vector of size $s$ with all entries being zero. \\
 \hline
 \lstinline|scalar_vector<T>(s, v)| & Vector of size $s$ with all entries equal to $v$. \\
 \hline
 \lstinline|random_vector<T>(s, d)| & Vector of size $s$ with all entries random according to the distribution specified by $d$. \\
 \hline
\end{tabular}
\end{center}
For example, to initialize a vector \lstinline|v1| with all $42$ entries being $42.0$, use
\begin{lstlisting}
 viennacl::vector<float> v1 = viennacl::scalar_vector<float>(42, 42.0f);
\end{lstlisting}

Similarly the following initializer types are available for matrices:
\begin{center}
\begin{tabular}{|l|p{10cm}|}
 \hline
 \lstinline|identity_matrix<T>(s, i)| & Identity matrix of dimension $s \times s$. \\
 \hline
 \lstinline|zero_matrix<T>(s1, s2)| & Matrix of size $s_1 \times s_2$ with all entries being zero. \\
 \hline
 \lstinline|scalar_matrix<T>(s1, s2, v)| & Matrix of size $s_1 \times s_2$ with all entries equal to $v$. \\
 \hline
 \lstinline|random_matrix<T>(s1, s2, d)| & Vector of size $s$ with all entries random according to the distribution specified by $d$. \\
 \hline
\end{tabular}
\end{center}

